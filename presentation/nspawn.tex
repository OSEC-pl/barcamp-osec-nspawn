%
% Copyright (c) 2016 Radoslaw Kujawa.
% All rights reserved.
%
% Redistribution and use in source and binary forms, with or without
% modification, are permitted provided that the following conditions
% are met:
%
% 1. Redistributions of source code must retain the above copyright
%    notice, this list of conditions and the following disclaimer.
% 2. Redistributions in binary form must reproduce the above copyright
%    notice, this list of conditions and the following disclaimer in the
%    documentation and/or other materials provided with the distribution.
%
% THIS SOFTWARE IS PROVIDED BY RADOSLAW KUJAWA (THE AUTHOR) AND CONTRIBUTORS
% ``AS IS'' AND ANY EXPRESS OR IMPLIED WARRANTIES, INCLUDING, BUT NOT LIMITED
% TO, THE IMPLIED WARRANTIES OF MERCHANTABILITY AND FITNESS FOR A PARTICULAR
% PURPOSE ARE DISCLAIMED.  IN NO EVENT SHALL THE AUTHOR OR CONTRIBUTORS
% BE LIABLE FOR ANY DIRECT, INDIRECT, INCIDENTAL, SPECIAL, EXEMPLARY, OR
% CONSEQUENTIAL DAMAGES (INCLUDING, BUT NOT LIMITED TO, PROCUREMENT OF
% SUBSTITUTE GOODS OR SERVICES; LOSS OF USE, DATA, OR PROFITS; OR BUSINESS
% INTERRUPTION) HOWEVER CAUSED AND ON ANY THEORY OF LIABILITY, WHETHER IN
% CONTRACT, STRICT LIABILITY, OR TORT (INCLUDING NEGLIGENCE OR OTHERWISE)
% ARISING IN ANY WAY OUT OF THE USE OF THIS SOFTWARE, EVEN IF ADVISED OF THE
% POSSIBILITY OF SUCH DAMAGE.
%
% 
\documentclass[dvipsnames,table]{beamer}
\usepackage{polski}

\usetheme{Rochester}
\usecolortheme{orchid}

\usepackage{listings}
\usepackage{ucs}
\usepackage[utf8x]{inputenc}
\usepackage{wasysym}
\usepackage[normalem]{ulem}
\usepackage{amsmath}
\usepackage{hyperref}
\usepackage{tikzsymbols}

\setbeamertemplate{navigation symbols}{}
\setbeamertemplate{caption}[numbered]
\setbeamerfont{caption}{size=\scriptsize}
\setbeamercolor{framenote}{bg=OSEC-red!25}
\setbeamercolor{rednote}{bg=Red!25}
\setbeamercolor{palette primary}{use=structure,fg=white,bg=OSEC-red}
\setbeamercolor{palette secondary}{use=structure,fg=white,bg=OSEC-red2}

\setbeamertemplate{itemize item}{\scriptsize\raise1pt\hbox{\donotcoloroutermaths$\blacktriangleright$}}
\setbeamertemplate{itemize subitem}{\tiny\raise1pt\hbox{\donotcoloroutermaths$\bullet$}}
\setbeamertemplate{itemize subsubitem}{\tiny\raise1pt\hbox{\donotcoloroutermaths{--}}}

\setbeamertemplate{enumerate item}{\insertenumlabel.}
\setbeamertemplate{enumerate subitem}{\insertenumlabel.\insertsubenumlabel}
\setbeamertemplate{enumerate subsubitem}{\insertenumlabel.\insertsubenumlabel.\insertsubsubenumlabel}
\setbeamertemplate{enumerate mini template}{\insertenumlabel}

\setbeamercolor{itemize item}{fg=OSEC-red, bg=OSEC-red}
\setbeamercolor{itemize subitem}{fg=OSEC-red, bg=OSEC-red}
\setbeamercolor{itemize subsubitem}{fg=OSEC-red, bg=OSEC-red}

\setbeamercolor{section number projected}{fg=white,bg=OSEC-red}
\setbeamercolor{subsection number projected}{fg=white,bg=OSEC-red}
\setbeamercolor{button}{bg=OSEC-red,fg=white}

\setbeamertemplate{section in toc}[circle]
\setbeamertemplate{subsection in toc}[square]

\definecolor{OSEC-red}{RGB}{160,29,44}
\definecolor{OSEC-red2}{RGB}{177,76,12}
\hypersetup{colorlinks=true,linkcolor=white,urlcolor=OSEC-red}

\setlength{\tabcolsep}{8pt}
\renewcommand{\arraystretch}{1.2}

%\lstset{
%	language=java,
%	basicstyle=\tiny,
%	breaklines=true,
%	escapechar=\@,
%	commentstyle=\color{OSEC-red}
%}

%\AtBeginSection[]{
%	\frame{
%
%		\begin{center}
%
%			{\usebeamerfont{section name}\usebeamercolor[fg]{section name}Część~\insertsectionnumber}
%			\vskip1em\par
%
%			\begin{beamercolorbox}[sep=12pt,center]{palette primary}
%				\usebeamerfont{section title}\insertsection\par
%			\end{beamercolorbox}
%		\end{center}
%		%\sectionpage
%	}
%}

\title{systemd-nspawn \\ zarządzanie kontenerami}
\author{Radosław Kujawa -- radoslaw.kujawa@osec.pl}
\institute{OSEC}

\begin{document}

\begin{frame}
	\titlepage
\end{frame}

\begin{frame}[allowframebreaks]
	\frametitle{Spis treści}
	{
		\hypersetup{colorlinks=true,linkcolor=black,urlcolor=OSEC-red}
		\tableofcontents
	}
\end{frame}


\begin{frame}
\frametitle{}
\begin{itemize}
	\item Ułatwienie procesu developmentu oraz testowania 
	\item Utrudnienie w procesie wdrożenia na produkcje
	\begin{itemize}
		\item Docker zorientowany na proces developmentu i testowania
	\end{itemize}
\end{itemize}
\end{frame}

\begin{frame}
\frametitle{Kontenery systemd}
\begin{itemize}
	\item Oparte o mechanizm namespace.
	\item Dobrze zintegrowane z innymi narzędziami systemd.
	\item Chronione SELinuxem.
	\item Korzystające z narzędzi hosta do instalacji kontenerów, a także zarządzania nimi.
\end{itemize}
\end{frame}

\begin{frame}
\frametitle{Instalacja}
\begin{itemize}
	\item Zalecane systemd 231 lub nowsze (Fedora 25).
	\item Paczka {\tt systemd-container} dostarcza potrzebne narzędzia.
\end{itemize}
\end{frame}

\begin{frame}
\frametitle{Instalacja kontenera}
\begin{itemize}
	\item Fedora: {\tt dnf --releasever=24 --installroot=/container/foo/ install systemd passwd dnf fedora-release}
	\item Debian: {\tt debootstrap --arch=amd64 stable /container/bar}
	\item {\tt semanage fcontext -a -t container\_file\_t '/container(/.*)?'}
	\item {\tt restorecon -R /container/}
\end{itemize}	
\end{frame}


\begin{frame}[fragile]
\frametitle{aaa}
\scriptsize
\begin{verbatim}
ooo oooooooo
oo
ooo
o
\end{verbatim}
\normalsize
\begin{itemize}
	\item foo {\tt bbb\_aaarr}.
\end{itemize}

\end{frame}

\begin{frame}
	\frametitle{Zarządzanie storage kontenera}
	\begin{itemize}
		\item System plików root - osobny dla każdego kontenera?
		\item Mount root ro
		\item btrfs
	\end{itemize}
\end{frame}

\begin{frame}
	\frametitle{Dostęp do katalogów hosta z kontenera}
	\begin{itemize}
		\item --bind
	\end{itemize}
\end{frame}

\begin{frame}
\frametitle{Alternatywa dla Dockera?}
\begin{itemize}
	\item Mechanizmy dostarczane przez systemd są bardziej niskopoziomowe.
	\item {\tt rkt} używa {\tt systemd-nspawn}.
\end{itemize}	
\end{frame}
\begin{frame}

\frametitle{Koniec\ldots}
\begin{center}
\includegraphics[scale=0.5]{img-oseclogo.png}

Dziękuje!

Czy są pytania?

\end{center}

\end{frame}
 
\end{document}
